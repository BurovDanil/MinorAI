\documentclass{article}

\usepackage{graphicx}
\usepackage{hyperref}

\begin{document}
	
	\begin{titlepage}
		\centering
		\vspace*{3cm}
		
		\includegraphics[width=0.4\textwidth]{images/fontyslogo.png} % Adjust the size of the logo
		\vspace{1.5cm}
		
		{\Huge\bfseries Personal Development Report}\\[1cm]
		{\LARGE Evaluation 2}\\[2cm]
		
		\textbf{Danil Burov}\\
		\vspace{0.5cm}
		\textit{Fontys University of Applied Sciences}\\[3cm]
		\vfill
		\textbf{\today}
		
	\end{titlepage}
	\newpage
	\tableofcontents
	\newpage
	
	\section{Introduction}
	My name is Danil Burov and I am studying Information Technology in Fontys, Venlo. I am specializing in 
	Embedded Software back in Venlo and the reason I chose this minor is because I think that AI is a very 
	ongoing topic and I wanted to get more familiar with it. Especially because I think that AI has a big 
	implication when it comes down to IoT devices.
	
	\section{Learning outcome 1 - Societal impact}
	\underline{\textbf{Description}}\\
	The student is able to approach the context and impact of their own AI project(s) 
	from different perspectives in a sustainable way. In addition, the student is able to 
	reflect on their own choices, taking into account data legislation and the (possible) impact on society.
	
	\underline{\textbf{Explanation:}}\\
	Societal impact is one of the most important contextual parts of the project. Every time a person 
	starts a project an evaluation on the societal impact should be done in order to understand 'Why?'
	a project is being done on the given topic.
	
	\subsection{First evaluation}
	During the first weeks of the minor I had the opportunity to think about the societal impact of the projects I will 
	be working on. Since the beginning I had an idea of what my personal project could look like. However, I was not that aware
	what could be the societal impact of my project. With the help of the technical coach 'Jacco' I established an understanding what
  could be a potential idea. The feedpulse submission will be shown under:\\
  \includegraphics[width=\textwidth,keepaspectratio]{images/Feedback_Societal_Impact.png}
	
	\underline{\textbf{Self assessment: Orientating}}
	
	\subsection{Second evaluation}
	From week six to week nine I did not really improve on my societal impact learning outcome. This is mainly due to the reason that 
	I was mainly focusing on modelling my first model and learning how to prepare data for training. In the upcoming weeks I will make sure that 
	I take time as well to work on the societal impact goal as well, starting with writing the 'Potential Impact Assessment'.
	
	\underline{\textbf{Self assessment: Orientating}}
	
	\subsection{Third evaluation}
	\subsection{Final evaluation}
	
	\section{Learning outcome 2 - Investigative problem solving}
	\underline{\textbf{Description}}\\
	The student is able to critically look at their own AI project(s) from different perspectives, 
	recognize problems and come up with appropriate solutions.
	
	\underline{\textbf{Explanation:}}\\
	From this learning outcome I will further develop my investigative and problem solving skills regarding 
	how AI is used in the project, is it doing as intended and if not how to approach the problem and solve it 
	eventually.
	
	\subsection{First evaluation}
	During the first weeks I had to tackle several problems. In the beginning, the course was presented with some stakeholders 
	who presented to us what issues they have with their companies and what they would like to add to their business. I was given 
	the opportunity to think of an idea that solves a business case and so I created a project proposal for one of the companies that visited us. 
	The proposal is based on the problem that the company has.
	
	To see the full proposal go to this link: \href{https://github.com/BurovDanil/MinorAI/blob/main/Documents/Project%20Proposal/Proposal.md}{project proposal}
	
	\underline{\textbf{Self assessment: Orientating}}
	
	\subsection{Second evaluation}
	I have been dealing with a legal issue for the past weeks because, for my personal project, I would love to have as much actual data as possible. 
	In order to do that, I needed to develop a scraper for the websites' statistics. Before doing so I checked the terms and conditions of the website. 
	Unfortunately, the website forbids any scraping and bots developed for the purpose of gathering data. So I investigated a way to still be able 
	to scrape the data and use it for my personal project. In the upcoming weeks I will create a document where I explicitly say what is the intended usage of 
	my project and why the scraper was developed.
	
	\underline{\textbf{Self assessment: Beginning}}
	
	\subsection{Third evaluation}
	\subsection{Final evaluation}

	\section{Learning outcome 3 - Data preparation}
	\underline{\textbf{Description}}\\
	The student is able to collect data and estimate its quality and usability. 
	The student is also able to adjust the data if necessary for proper usage in their project(s).
	
	\underline{\textbf{Explanation:}}\\
	In order to create any AI model I will need a good, clean dataset that can be used to properly teach a machine.
	To do so I intend to first of all use a reliable source for the set and inspect it carefully before using.
	
	\subsection{First evaluation}
	The first thing that was done regarding this learning outcome was to select a dataset suitable for the intended usage of the 
	AI model. For the personal project I have selected this \href{https://www.kaggle.com/datasets/asaniczka/ufc-fighters-statistics}{dataset}. I have been working on a cleaned version of this dataset. 
	Since this dataset does have some empty fields I tried already cleaning some of the data to prepare it for the upcoming modeling. Regarding the data preparation 
	for the project, I have already started analyzing the data for the organization we are doing the project for, however it is still in a very early stage.
	
	\underline{\textbf{Self assessment: Orientating}}
	
	\subsection{Second evaluation}
	I have been preparing data for the past couple of weeks for both my personal and group projects. In the past weeks in the group project, I have been 
	trying to estimate the quality and the usability of the data that was provided by our stakeholder. To assess the quality and usability I cleaned 
	the data from any 'null' fields and factorized some of the features. I also combined features into a new feature. Especially in my personal 
	project, I had to check each fighters' statistics and based on them I labeled the fighters' fight style.
	
	\underline{\textbf{Self assessment: Beginning}}
	
	\subsection{Third evaluation}
	\subsection{Final evaluation}
	
	\section{Learning outcome 4 - Machine teaching}
	\underline{\textbf{Description}}\\
	The student is able to use data to train models in a way that fits the intended purpose.
	The student is also able to test whether the models have been adequately trained.
	
	\underline{\textbf{Explanation:}}\\
	After making sure that the dataset that will be used to teach a machine is not corrupted, outdated, etc., 
	I need to ensure that the machine is doing as intended with the dataset. In order to ensure quality, testing the
	machine teaching is ideal.
	
	\subsection{First evaluation}
	In the first weeks of the minor I explored a website called 'Teachable Machine' in order to get familiar with the steps of creating an 
	AI model. The usage of my model was to recognize cats and dogs.Other than using the website to get familiar with machine learning, 
  I have tried to familiarize myself with different machine learning algorithms, such as Decision Trees, Random Forest, etc. Here you can see the received feedback from one
  of my lectures:\\
  \includegraphics[width=\textwidth,keepaspectratio]{images/FeedbackTeachableMachines.png}

  \underline{\textbf{Self assessment: Orientating}}

	\subsection{Second evaluation}
	I managed to model the data of my personal project, whereas in my group project I mainly analyzed and cleaned the data. The difficulties
	I found along the way were mainly choosing the model that was actually fitting the purpose of my project. I had to choose between a 'Classifier'
	and a 'Regressor' first. For my personal project, I chose to use the 'Random Forest Classifier' model, because I was mainly trying to have a 
	prediction of yes or no. However, after my meeting with my semester coach Bas I decided to redirect my focus on actually estimating 
	the likelihood of each fighter winning a match against each other. The upcoming weeks I will be trying to teach a model with the data I have
  analyzed in my project.
	
  \underline{\textbf{Self assessment: Beginning}}
	
	\subsection{Third evaluation}
	\subsection{Final evaluation}
\section{Learning outcome 5 - Data visualization}
\underline{\textbf{Description}}\\
The student is able to use data to create an interesting, informative and 
compelling story in an (interactive) data visualization product, tailored to the right target group.\\
\underline{\textbf{Explanation:}}\\
Visualizing data will help further understand the relation between different features. In order to achieve this goal,
every correlation found between different features needs to be visualized. These images need to be understandable and 
self-explanatory.

\subsection{First evaluation}
I have not yet done anything related to data visualization. I have only educated myself with the provided Python tutorials and self-learning.\\
\underline{\textbf{Self assessment: Orientating}}

\subsection{Second evaluation}
I have been able to visualize the correlations with the features for my project. Unfortunately, the correlations are not that strong and I fear that 
the data is very unaccurate and that is making the data show this information. In the future I would like to find correlations for my personal project, answering 
the question which fighter statistics make the percentage of a fighters' chance of winning higher. \\\\
Here you can see the correlation matrix I created for my group project:

\includegraphics[width=\textwidth,keepaspectratio]{images/Stagri_correlation_matrix.png}
\underline{\textbf{Self assessment: Beginning}}

\subsection{Third evaluation}
\subsection{Final evaluation}

\section{Learning outcome 6 - Reporting}
\underline{\textbf{Description}}\\
The student is able to report in a methodologically sound manner on (the outcome of) 
their own AI projects (project proposal, process documentation, reporting of final results, etc.).\\
\underline{\textbf{Explanation:}}\\
It is important that documents are written in time and feedback is used to prove the legitimacy of the given goal. The goal would be considered 
accomplished if all documents are consistent and comprehensible. It is very important that the code documentation is easy
to understand and use.

\subsection{First evaluation}
It is essential for me to document the progress I make in order to keep track of how I have improved during the minor. To do so, 
I have created my own personal repository where I have already put all the documents I have written up until now, as well as all the exercises 
I have completed over the past few weeks.\\\\
\underline{\textbf{Self assessment: Beginning}}

\subsection{Second evaluation}
I have added markdown 'README' file in my repository to make everything more structured and to be easily found. I am trying to work as transparently as
possible and document the work I do. In order to keep it transparent I write markdown in the python notebooks,ensuring that everything I do is 
understandable and comprehensive.\\\\
Link to my personal repository: \href{https://github.com/BurovDanil/MinorAI}{Personal repository for the minor}\\
Link to my personal project notebook: \href{https://github.com/BurovDanil/MinorAI/blob/main/PythonCR/Personal%20project/UfcModel.ipynb}{Personal project notebook}\\
Link to my group project notebook: \href{https://github.com/AI-Farming-Thewi/cows_analysis/blob/stagri-farm-analysis/Stagri%20analysis.ipynb}{Group project notebook}\\\\
\underline{\textbf{Self assessment: Proficient}}

\subsection{Third evaluation}
\subsection{Final evaluation}

\section{Learning outcome 7 - Personal Leadership}
\underline{\textbf{Description}}\\
The student shows an entrepreneurial mindset regarding their own AI project(s) and personal development, while being aware
of their own learning capacity and keeping in mind professional ambitions in their future work field.\\
\underline{\textbf{Explanation:}}\\
The outcome of this learning goal should be that I manage my time correctly without overloading myself. That means following some kind of a schedule, 
as well as attending lectures to ensure that I stay on track and seeking feedback to make sure I am progressing.

\subsection{First evaluation}
Ever since the minor started, I have attended all lectures regarding the minor and have taken some insights that will help me further develop my skills in AI modeling. 
I still need to improve on asking for more feedback from lecturers.
\underline{\textbf{Self assessment: Orientating}}

\subsection{Second evaluation}
I have been following a sort of strict schedule when it comes down to work. I am working from the morning to the late afternoon every Tuesday and Thursday
with my group. That way I ensure that I am always up-to-date with what my teammates are doing in the project. Moreover, I do my personal project work on Monday and Wednesday, leaving 
Friday to be a day which is free. This was done intentionally, making sure that if I feel that something needs more work I can do it then. \\\\
\underline{\textbf{Self assessment: Proficient}}
\subsection{Third evaluation}
\subsection{Final evaluation}

\section{Learning outcome 8 - Personal goal}
\underline{\textbf{Description}}\\
With this learning outcome, the student can set their own goal in relation to their future field of work.
This is always related to your Individual Challenge. Describe this Learning Outcome in your PDR.

\underline{\textbf{Explanation:}}\\
My personal goal for this minor is to become very familiar with how to create an AI model, be able to analyze large amounts of data,
and prepare this data for further training of a given model. I would like to get more familiar with how machine learning algorithms work 
and be able to choose the appropriate algorithm for the use case.

\subsection{First evaluation}
In the first 4 weeks, I managed to choose a dataset for my personal project that fits my needs. I attended all lectures to make 
sure I progress in my knowledge of AI, both technically and ethically. I managed to import the dataset and clean it based on a condition.

\underline{\textbf{Self assessment: Orientating}}

\subsection{Second evaluation}
From week six to nine I believe that I have progressed quite a lot since the previous four weeks. I have done a lot of data preparation for 
modelling and even managed to train my first model. I am very content with the work I have done regarding my personal project, tackling legal issues 
as well battling accuracy in the moment. I know that the current dataset may not be enough to train a model sufficiently as well, so I am very excited for the 
upcoming weeks and how both my projects will progress. I am currently getting familiar with different machine learning algorithms since I found not much 
success in my personal projects' model I am currently using.
\subsection{Third evaluation}
\subsection{Final evaluation}

\section{Retrospect} % ONLY IN FINAL VERSION

\section{Conclusion} % ONLY IN FINAL VERSION

\end{document}
