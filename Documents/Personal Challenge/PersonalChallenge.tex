\documentclass{article}
\usepackage{hyperref}
\title{Personal Challenge}
\author{Danil Burov}
\date{\today}

\begin{document}

\maketitle

\section{Proposal}

My personal challenge will be to create an AI model that predicts how a match between two
MMA fighters will end up. Which round is the fight going to end, via what kind of stoppage. 
The dataset I will use will be \href{https://www.kaggle.com/datasets/asaniczka/ufc-fighters-statistics}{UFC dataset}.
UFC is the biggest MMA organization in the world where the best of the best fighters strive to
compete. I chose the dataset from the UFC and not some other organization because I do follow UFC
and there is a lot of data regarding fighter statistics.

\section{Societal impact}
Using this model a fighter could evaluate what could be the reason the other fighter has a bigger
chance of winning. My intention is eventually create an analytical tool that analyzes why a fighter 
was defeated and what could be the possible solution for him to win the fight if they ever fight again.

\section{Investigative problem solving}
There are a few problems that might occur with this project. One of them being the illegal usage of the
AI model. My suspicion is that in the wrong hands the model could be use to bet on specific matches. In order 
to address this issue I was advised to eventually write 'Disclaimer' which specifies the intended usage of the model. 
Another potential solution would be to make the model intentionally less accurate to ensure that it cannot be used for betting.

\section{Data preparation}
As mentioned above I have already selected the dataset for the project. However, this dataset will not be copy-pasted into 
the model. I will need to make sure that only active fighters are actually in the dataset to ensure that the data is clear. 
As well as there are some 'null' values in the dataset and they will be cleared and formatted to ensure consistency.

\section{Machine teaching}
The dataset will be used to teach a machine to predict the outcome of a fight. In order to test if the model 
is adequetly trained I will test it with existing fights that have already happened and see if the model does predict the outcome. This
will be possible because I will not teach the model the history of all the UFC matches that have already happened but just fighter statistics.

\section{Data visualization}
In my personal project I will visualize correlations between different features. For example: "Does the win-lose ration of a fighter based 
on their fightstyle or their age?, If you are a specific type of fighter do you have a higher percentage to win?".

\section{Reporting}
In order to have consistent documentation in  my personal project everything will be document in my personal Github repository. In order to ensure 
transparency and that every person can access the repository it will be made 'public'. A link to the \href{https://github.com/BurovDanil/MinorAI}{repository}

\section{Personal goal}
My personal goal is to become comfortable and familiar with preparing data for modelling, including tasks such as cleaning, transforming,
and structuring data to ensure that is can be used for machine learning algorithms. Involving handling missing values, converting data types.

\end{document}
